%%----------------------------------------------------------%%
%% Abteilungsheader setzen!                                 %%
%% Dazu richtige Zeile aktivieren (d.h. % entfernen) und    %%
%% andere Zeile auskommentieren (d.h. % am Zeilenanfang     %%
%% setzen)                                                  %%
%%                                                          %%
%%  Bitte \ und folgenden Text nicht entfernen,             %%
%%  das ist ein LaTeX-Befehl                                %%
%%  Bitte %% und folgenden Text nicht entfernen,            %%
%%  das ist eine Erkl"arung des Befehls                     %%
%%----------------------------------------------------------%%

\ifs     %% Logo der Abt. f"ur Mathematische Stochastik

\btabt
\newcommand{\seminarlesekurs}[1]{%
	\kopfeintrag{Seminar/Lesekurs}{#1}{#1}
}
\seminarlesekurs{Shape Analysis} 
\dozent{Philipp Harms}
\zeitort{Mi 14--16 Uhr, Raum 232, Ernst-Zermelo-Stra\ss{}e~1}
\vorbesprechung{Mi 17.~Okt.~2018, 14:15 Uhr, selber Ort}
\teilnehmerliste{Um teilzunehmen, kommen Sie bitte in die Vorbesprechung des Seminars; eine Teilnehmerliste wird nicht vorab ausliegen.} %% Ende teilnehmerliste
%\fragestunde{wer, Wochentag, Uhrzeit, Raum und Straße}
\webseite{\url{https://www.stochastik.uni-freiburg.de/lehre/ws-2018-2019/seminar-shapeanalysis-ws-2018-2019/info-seminar-shapeanalysis-ws-2018-2019}}
\etabt

%%--------------------------------------------------------------%%
%% Beschreibung des Inhalts der Veranstaltung                   %%
%% Inhalt in die geschweiften Klammern einfuegen                %%
%%                                                              %%
%%  Das Wort  "Inhalt" kommt automatisch                        %%
%%--------------------------------------------------------------%%

\inhalt{%%
Shape Analysis beschäftigt sich mit der Modellierung und Analyse von geometrischen Daten. 
Beispielsweise sind dies Datensätze von Kurven, Flächen und Tensorfeldern aus bildgebenden Verfahren der Medizin, oder Bilddaten mit Tiefeninformation, die von einigen Handykameras bereits mitgeliefert wird. 
Shape Analysis ist ein interdisziplinäres Forschungsgebiet, welches Methoden und Fragestellungen aus folgenden Gebieten vereint:
\begin{itemize}
\item Riemannsche Differentialgeometrie in endlicher und unendlicher Dimension
\item Statistik, Stochastik und Machine Learning auf Mannigfaltigkeiten
\item Anwendungen in Computational Anatomy, Computergrafik, Anthropologie und weiteren Gebieten mit nichtlinearen hochdimensionalen Daten
\end{itemize}
Die Themen des Seminars werden je nach Vorwissen und Interesse ausgewählt. 
Geplant ist eine Einführung in differentialgeometrische Aspekte von Shape Analysis, gefolgt von individuellen Einheiten zu angewandteren Themen.
}   %% Klammer bitte stehen lassen, Ende von \inhalt{


%%--------------------------------------------------------------%%
%% Literaturverzeichnis zur Veranstaltung                       %%
%% Inhalt hinter `\item' in die geschweiften Klammern einf"ugen.%% 
%% Weiterer Eintrag durch `\item{xxx}' in neuer Zeile           %%
%%--------------------------------------------------------------%%
%% Das Wort "Literatur" erscheint automatisch                   %%
%%------------------------------------------------------------- %%
%% !! Von dieser Literatur sollte in der Bibliothek mehr     !! %%
%% !! als ein Exemplar vorhanden sein, damit sie f\"ur den     !! %%
%% !! Semesterapparat gesperrt werden kann!!!                !! %%
%%--------------------------------------------------------------%%

%\blit
%
%\elit

%%--------------------------------------------------------------%%
%% Fusszeilen mit zus"atzlichen Infos zur Veranstaltung         %%
%%--------------------------------------------------------------%%
%% Inhalt bitte in die geschweiften Klammern einfuegen          %%
%%                                                              %%
%% Die Worte "Typisches Semester", "Punkte",                    %%
%% "Nuetzliche Vorkenntnisse", "Sprechstunde Dozent(in)" etc.   %%
%% erscheinen automatisch                                       %%
%%--------------------------------------------------------------%%


\btabb
%
%\ectspunkte{9} % !! das Wort "Punkte" kommt automatisch; bitte ECTS-Punkte angeben
%\typischessemester{ab dem 2. Semester im Master}
%\verwendbarkeit{ \textcolor{red}{--- Bitte ausw\"ahlen: ---} Angewandte Mathematik; Reine Mathematik; Kategorie II; Kategorie III }%
\notwendigevorkenntnisse{Elementare Differentialgeometrie}
%\nuetzlichevorkenntnisse{ --- Bitte angeben oder auskommentieren --- }
%\folgeveranstaltungen{ --- Bitte angeben oder auskommentieren --- }
\studienpruefungsleistung % !! stehen lassen !!
%\zusatzbemerkung{ --- Bitte angeben oder auskommentieren --- }
%
%
\etabb
\clearpage
